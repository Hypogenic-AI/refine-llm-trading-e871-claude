\section{Conclusion}
\label{sec:conclusion}

We presented the first controlled experiment isolating decision frequency as the independent variable for LLM trading agents.
Testing \gptmini across daily, weekly, and monthly rebalancing on five major U.S.\ stocks over 2023--2024, we identified a non-monotonic ``Goldilocks zone'': weekly rebalancing achieves a 15\% higher Sharpe ratio than daily (1.028 vs.\ 0.892) while requiring 75\% fewer trades, whereas monthly rebalancing degrades performance by 53\% (Sharpe 0.421).

Our results carry a practical message for the LLM trading community: \emph{default to weekly, not daily, rebalancing.}
This simple change improves risk-adjusted returns, reduces API costs by approximately 80\%, and works robustly across 4 of 5 tested stocks.
Theoretically, the finding suggests that LLMs have a temporal sweet spot for financial reasoning---daily is too noisy, monthly too coarse---that aligns with their strength in semantic pattern recognition over multi-day horizons.

Important caveats remain.
No LLM configuration outperforms \buyhold in this bull market, consistent with \finsaber \citep{li2025finsaber}.
The test period, stock universe, and model are all limited.
Future work should extend to bear markets, larger stock universes, multiple LLM backbones, and hybrid architectures that combine weekly LLM strategic decisions with daily rule-based execution.
The question of whether the weekly optimum reflects a fundamental property of LLM reasoning or an artifact of the 5-day trading week structure remains open and warrants further investigation.
