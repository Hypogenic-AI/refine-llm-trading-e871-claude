\section{Discussion}
\label{sec:discussion}

\subsection{Why Weekly Works Best}
\label{sec:why_weekly}

Our results reveal a non-monotonic relationship between decision frequency and LLM trading performance.
We offer three complementary explanations for why weekly rebalancing sits at the optimum.

\para{Noise filtering.}
Daily price movements contain substantial noise from intraday trading, market microstructure, and short-term sentiment fluctuations.
The LLM agent at daily frequency reacts to this noise, leading to excessive trading (127 trades on average) and frequent whipsawing between positions.
Weekly aggregation smooths out daily fluctuations, allowing the agent to focus on genuine 5-day trends.
This aligns with the finding from \finpos that daily rewards cause ``overreaction to noise'' \citep{liu2025finpos}.

\para{Signal preservation.}
At monthly frequency, the agent makes only 24 decisions over two years---too few to respond to regime changes, earnings surprises, or multi-week rallies.
TSLA and NFLX both lost money at monthly frequency because the agent missed significant price movements between decision points.
The weekly cadence provides enough decision points (100 over 2 years) to capture major trends without drowning in daily noise.

\para{Alignment with LLM reasoning.}
LLMs process information through token-level pattern matching and semantic reasoning, not through precise numerical optimization.
A one-week horizon maps naturally to the kind of qualitative reasoning LLMs perform well: ``the stock has risen 5\% this week on strong earnings; the trend is likely to continue'' is a judgment LLMs can make more reliably than ``the stock dropped 0.3\% today; it will likely rebound tomorrow.''
This interpretation is consistent with \citet{darmanin2025lmrl}, who argue that LLMs are best suited for longer-term strategic direction rather than short-term tactical execution.

\subsection{Relationship to Prior Work}
\label{sec:prior_work_discussion}

Our findings are consistent with, and extend, several recent results.
\finsaber \citep{li2025finsaber} found that LLM agents are ``too conservative in bull markets and too aggressive in bear markets'' at the daily frequency.
We observe the same conservatism: the daily LLM agent achieves only +53.8\% return during a period where \buyhold returns +162.2\%.
Weekly rebalancing partially addresses this, boosting returns to +62.4\%, but the gap remains large.

\finpos \citep{liu2025finpos} demonstrated that 30-day reward horizons outperform shorter windows for their dual-agent architecture.
Our finding that the optimal \emph{decision} frequency is weekly (5 days) rather than monthly (21 days) suggests that the optimal horizon may differ depending on whether it governs the reward computation or the decision timing.
When the agent can still make daily micro-adjustments (as in \finpos), longer reward horizons help.
When the agent can only act at the specified frequency (as in our setup), weekly provides a better balance.

\subsection{Limitations}
\label{sec:limitations}

We identify several limitations that qualify our findings.

\para{Bull market bias.}
Our 2023--2024 test period was predominantly bullish, with all five stocks showing strong positive returns.
The weekly advantage may behave differently in bear or sideways markets.
Extending the analysis to include 2020 (COVID crash and recovery) and 2022 (bear market) is an important next step.

\para{Small stock universe.}
With only $n = 5$ stocks, our paired statistical tests lack power: the medium effect sizes ($d = 0.59$ for weekly vs.\ daily, $d = -0.66$ for monthly vs.\ daily) would likely reach significance with $n = 15$--20 stocks.
All five stocks are large-cap technology companies, and the results may not generalize to small-caps, value stocks, or other sectors.

\para{Single model.}
We test only \gptmini.
Different LLM architectures (GPT-4.1, Claude, Gemini) may exhibit different horizon sensitivities.
Models with stronger numerical reasoning might perform better at daily frequency, while models with stronger narrative reasoning might benefit even more from weekly.

\para{Price-only signals.}
The agent receives only price data and simple technical indicators.
Adding news sentiment, earnings data, or SEC filings could change the optimal horizon---news events are inherently discrete and may favor decision points aligned with information release schedules.

\para{No position sizing or short selling.}
The binary long/cash position limits the agent's expressiveness.
Partial positions, portfolio-level allocation, and short selling would provide a more realistic trading environment and could interact with the frequency variable.

\para{Near-deterministic monthly results.}
With only 24 decision points and temperature 0.3, the monthly LLM agent produces nearly identical decisions across seeds.
This means monthly ``variance'' is effectively zero, and poor monthly performance reflects the agent's systematic behavior rather than stochastic noise.
