\section{Methodology}
\label{sec:method}

We design a controlled experiment where the \emph{only} independent variable is the decision frequency of the LLM trading agent.
The agent architecture, prompt template, LLM backbone, data sources, and evaluation metrics are held constant across all conditions.

\subsection{Data}
\label{sec:data}

We use daily OHLCV (Open, High, Low, Close, Volume) data from Yahoo Finance for five major U.S.\ stocks: AAPL, MSFT, AMZN, TSLA, and NFLX.
The test period spans January 2, 2023 to December 31, 2024, covering 502 trading days.
All prices are split-adjusted and contain no missing values.
This period was predominantly bullish, with all five stocks showing positive Buy-and-Hold returns ranging from +78.6\% (MSFT) to +273.2\% (TSLA).

\subsection{LLM Trading Agent}
\label{sec:agent}

At each decision point, the agent receives a structured prompt containing:
(1) the current date and stock ticker;
(2) the current price and position status (cash or long);
(3) the most recent 10 closing prices;
(4) 5-day and 20-day simple moving averages (SMAs);
(5) price changes over 1-day, 5-day, 10-day, and 20-day windows; and
(6) the decision horizon (next day, next week, or next month).
The agent responds with exactly one word: \texttt{BUY}, \texttt{SELL}, or \texttt{HOLD}.
The prompt explicitly instructs the model to consider the appropriate time horizon.

We use \gptmini (\texttt{gpt-4.1-mini-2025-04-14}) as the LLM backbone with temperature 0.3 and a maximum of 5 output tokens.
Each stock begins with \$10{,}000 in capital and incurs a 0.1\% transaction cost per trade.
The agent can go fully long (invest all capital) or fully cash; partial positions and short selling are not supported.

\subsection{Rebalancing Frequencies}
\label{sec:frequencies}

We test three decision frequencies:
\begin{itemize}[leftmargin=*,itemsep=0pt,topsep=0pt]
\item \textbf{Daily}: the agent makes a decision every trading day (502 decision points over 2 years).
\item \textbf{Weekly}: the agent makes a decision every 5 trading days (100 decision points).
\item \textbf{Monthly}: the agent makes a decision every 21 trading days (24 decision points).
\end{itemize}
Between decision points, the agent's position remains unchanged.

\subsection{Baselines}
\label{sec:baselines}

We compare the LLM agent against three baselines:
\begin{itemize}[leftmargin=*,itemsep=0pt,topsep=0pt]
\item \textbf{\buyhold}: purchase on day 1 and hold throughout the entire period.
\item \textbf{\smacross}: a 20/50-day SMA crossover strategy that buys when the short SMA crosses above the long SMA and sells on the reverse.
\item \textbf{\randombl}: randomly selects BUY, SELL, or HOLD at each decision point.
\end{itemize}

\subsection{Evaluation Metrics}
\label{sec:metrics}

We report six metrics for each strategy:
\begin{itemize}[leftmargin=*,itemsep=0pt,topsep=0pt]
\item \textbf{Cumulative Return (CR\%)}: total return over the 2-year period.
\item \textbf{Sharpe Ratio (SR)}: annualized risk-adjusted return with risk-free rate $r_f = 0$.
\item \textbf{Maximum Drawdown (MDD\%)}: worst peak-to-trough decline during the period.
\item \textbf{Sortino Ratio}: return per unit of downside risk.
\item \textbf{Number of Trades}: total buy and sell actions executed.
\item \textbf{Win Rate}: fraction of decision periods producing positive returns.
\end{itemize}

The Sharpe ratio is our primary metric for hypothesis testing, as it captures both return magnitude and risk.

\subsection{Experimental Protocol}
\label{sec:protocol}

To account for stochasticity in LLM outputs, we run each configuration multiple times with different random seeds (42, 49, 56).
Daily configurations use 2 runs (lower inherent variance due to more decision points), while weekly and monthly configurations use 3 runs.
Results are averaged across runs and across the five stocks unless otherwise noted.

In total, the experiment involves approximately 9{,}375 API calls, costs \$0.82, and runs in 43.5 minutes.
We use paired $t$-tests across the five stocks to compare Sharpe ratios between frequencies and report Cohen's $d$ effect sizes.
