\section{Related Work}
\label{sec:related}

\para{LLM trading agents.}
The use of LLMs for financial trading has grown rapidly since GPT-3 \citep{brown2020gpt3} demonstrated strong in-context learning.
\finmem \citep{yu2023finmem} introduces a layered memory architecture with decay mechanisms for trading individual stocks.
\tradingagents \citep{xiao2024tradingagents} mimics institutional trading firms with specialized analyst, researcher, trader, and risk management agents.
\fincon \citep{yu2024fincon} proposes a manager-analyst hierarchy with conceptual verbal reinforcement.
\flagtrader \citep{xiong2025flagtrader} fuses LLM reasoning with reinforcement learning by using the LLM as a policy network fine-tuned via trading reward gradients.
A common thread across these systems is the reliance on \emph{daily} decision cycles, where the agent makes a single buy/sell/hold decision each trading day.
Our work departs from this convention by systematically varying the decision frequency.

\para{Benchmarking LLM trading.}
Several recent studies have rigorously evaluated LLM trading agents.
\finsaber \citep{li2025finsaber}, accepted at KDD 2026, benchmarks LLM strategies over 20 years using 100+ S\&P 500 constituents (including delisted stocks) and finds that neither \finmem nor \finagent generates statistically significant alpha.
\stockbench \citep{chen2025stockbench} reaches similar conclusions: ``excelling at static financial knowledge tasks does not necessarily translate into successful trading strategies.''
\citet{fan2025aitrader} test across U.S.\ stocks, A-shares, and cryptocurrency, finding that ``general intelligence does not automatically translate to effective trading capability.''
\citet{li2024investorbench} provide a standardized benchmark across stocks, crypto, and ETFs with 13 LLM backbones.
These benchmarking studies consistently find that LLM agents underperform simple baselines at the daily frequency.
Our work investigates whether \emph{changing} the decision frequency can narrow this performance gap.

\para{Temporal horizons in LLM trading.}
The most directly relevant work to ours is \finpos \citep{liu2025finpos}, which introduces position-aware trading with multi-timescale reward design.
\finpos evaluates rewards across 1-day, 7-day, and 30-day horizons and finds that the 30-day horizon produces the best performance, dramatically outperforming single-step daily approaches.
The authors argue that ``LLMs excel at extracting long-term trends and underlying causal structures from complex semantic information rather than performing high-frequency precise numerical optimization.''
However, \finpos varies the \emph{reward} horizon within a multi-agent architecture that still makes daily decisions.
Our work complements \finpos by varying the \emph{decision} frequency itself---the agent only acts at weekly or monthly intervals---providing a cleaner isolation of the temporal variable.

\para{High-frequency LLM trading.}
At the opposite end of the spectrum, \quantagent \citep{xiong2025quantagent} applies multi-agent LLMs to high-frequency trading on 1-hour and 4-hour bars, achieving up to 80\% directional accuracy.
Notably, \quantagent operates solely on price-derived signals, explicitly avoiding textual data which ``typically lags price discovery.''
\citet{darmanin2025lmrl} propose a hybrid framework where LLMs generate high-level strategies to guide RL agents for short-term execution, arguing that LLMs are best suited for longer-term strategic direction.
These works suggest a natural division of labor: LLMs for strategic, longer-horizon reasoning, and specialized models or rules for tactical execution.

\para{Surveys.}
\citet{ding2024llmagent} survey 51 LLM trading agent papers, providing a taxonomy of approaches including news-driven, reflection-driven, debate-driven, and RL-driven agents.
\citet{nie2024survey} cover financial LLM applications broadly, spanning sentiment analysis, time series forecasting, and agent-based trading.
Both surveys note the predominance of daily trading cycles and the difficulty of outperforming passive baselines---precisely the challenge our work addresses through the lens of decision frequency.
