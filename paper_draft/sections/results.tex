\section{Results}
\label{sec:results}

\subsection{Main Results: Weekly Rebalancing is Optimal}
\label{sec:main_results}

\tabref{tab:main_results} presents the aggregate performance of the LLM agent across the three rebalancing frequencies, averaged over five stocks and multiple runs.
Weekly rebalancing achieves the highest Sharpe ratio (1.028), outperforming daily (0.892) by 15\% and monthly (0.421) by 144\%.
Weekly also delivers the highest cumulative return (+62.4\%) while executing 75\% fewer trades than daily (31 vs.\ 127).
Monthly rebalancing substantially degrades performance across all metrics, with cumulative return dropping to +13.5\% and the Sortino ratio falling to 0.523.

\begin{table}[t]
    \centering
    \caption{LLM agent performance by rebalancing frequency, averaged across five stocks. Best LLM results in \textbf{bold}. Weekly achieves the best risk-adjusted returns while requiring far fewer trades.}
    \label{tab:main_results}
    \begin{tabular}{@{}lccccc@{}}
        \toprule
        \textbf{Frequency} & \textbf{CR (\%)} & \textbf{Sharpe} & \textbf{MDD (\%)} & \textbf{Sortino} & \textbf{Trades} \\
        \midrule
        Daily   & +53.8 & 0.892 & \textbf{$-$20.7} & 1.140 & 127 \\
        Weekly  & \textbf{+62.4} & \textbf{1.028} & $-$24.2 & \textbf{1.292} & \textbf{31} \\
        Monthly & +13.5 & 0.421 & $-$28.0 & 0.523 & 11 \\
        \bottomrule
    \end{tabular}
\end{table}

\subsection{Comparison to Baselines}
\label{sec:baseline_comparison}

\tabref{tab:baselines} compares all strategies.
No LLM configuration outperforms \buyhold (Sharpe 1.620), consistent with findings from \finsaber \citep{li2025finsaber}.
However, the weekly LLM agent approaches \smacross performance (Sharpe 1.028 vs.\ 1.058--1.099) and substantially outperforms \randombl (Sharpe 0.701).
The daily LLM agent underperforms \smacross (0.892 vs.\ 1.093), while the monthly LLM agent performs barely better than \randombl.

\begin{table}[t]
    \centering
    \caption{Comparison of all strategies averaged across five stocks. \buyhold dominates in this bull market, but weekly LLM rebalancing approaches \smacross performance.}
    \label{tab:baselines}
    \resizebox{\textwidth}{!}{%
    \begin{tabular}{@{}llcccc@{}}
        \toprule
        \textbf{Strategy} & \textbf{Frequency} & \textbf{CR (\%)} & \textbf{Sharpe} & \textbf{MDD (\%)} & \textbf{Trades} \\
        \midrule
        \buyhold         & ---     & \textbf{+162.2} & \textbf{1.620} & $-$26.2 & 1 \\
        \smacross        & Daily   & +61.2  & 1.093 & $-$24.2 & 9 \\
        \smacross        & Weekly  & +56.0  & 1.058 & $-$24.4 & 9 \\
        \smacross        & Monthly & +61.8  & 1.099 & $-$22.9 & 7 \\
        \randombl        & Daily   & +29.4  & 0.701 & $-$25.6 & 173 \\
        \midrule
        LLM (\gptmini)  & Daily   & +53.8  & 0.892 & \textbf{$-$20.7} & 127 \\
        LLM (\gptmini)  & Weekly  & +62.4  & 1.028 & $-$24.2 & 31 \\
        LLM (\gptmini)  & Monthly & +13.5  & 0.421 & $-$28.0 & 11 \\
        \bottomrule
    \end{tabular}%
    }
\end{table}

\subsection{Per-Stock Analysis}
\label{sec:per_stock}

\tabref{tab:per_stock} presents per-stock results, and \figref{fig:per_stock_sharpe} visualizes the Sharpe ratios.
The weekly advantage is robust: it holds for 4 of 5 stocks (AAPL, MSFT, AMZN, NFLX).
TSLA is the sole exception, where daily slightly outperforms weekly (Sharpe 1.213 vs.\ 1.100), likely due to TSLA's high volatility creating profitable short-term trading opportunities.

\begin{table}[t]
    \centering
    \caption{Per-stock LLM agent performance across rebalancing frequencies. Best Sharpe per stock in \textbf{bold}. Weekly outperforms daily in 4 of 5 stocks.}
    \label{tab:per_stock}
    \resizebox{\textwidth}{!}{%
    \begin{tabular}{@{}llcccccc@{}}
        \toprule
        & & \multicolumn{2}{c}{\textbf{Daily}} & \multicolumn{2}{c}{\textbf{Weekly}} & \multicolumn{2}{c}{\textbf{Monthly}} \\
        \cmidrule(lr){3-4} \cmidrule(lr){5-6} \cmidrule(lr){7-8}
        \textbf{Stock} & \textbf{B\&H SR} & \textbf{CR (\%)} & \textbf{Sharpe} & \textbf{CR (\%)} & \textbf{Sharpe} & \textbf{CR (\%)} & \textbf{Sharpe} \\
        \midrule
        AAPL & 2.04 & +40.9 & 1.163 & +49.0 & \textbf{1.304} & +18.0 & 0.584 \\
        MSFT & 1.62 & +7.6  & 0.284 & +29.4 & \textbf{0.781} & +23.1 & 0.654 \\
        AMZN & 2.10 & +21.5 & 0.531 & +30.1 & 0.678 & +34.5 & \textbf{0.684} \\
        TSLA & 1.14 & +129.9 & \textbf{1.213} & +121.7 & 1.100 & $-$6.8 & 0.102 \\
        NFLX & 1.97 & +69.2 & 1.270 & +81.8 & \textbf{1.276} & $-$1.5 & 0.080 \\
        \bottomrule
    \end{tabular}%
    }
\end{table}

\begin{figure}[t]
    \centering
    \includegraphics[width=0.95\linewidth]{figures/llm_per_stock_sharpe.png}
    \caption{Sharpe ratio by stock and rebalancing frequency. Weekly outperforms daily for 4 of 5 stocks, with the largest improvement for MSFT (+175\%). Monthly degrades sharply for volatile stocks (TSLA, NFLX).}
    \label{fig:per_stock_sharpe}
\end{figure}

The most dramatic improvement occurs for MSFT, where the daily Sharpe of 0.284 jumps to 0.781 at weekly frequency---a 175\% improvement.
MSFT has lower volatility than the other stocks, making daily price noise particularly unhelpful for the LLM.
AMZN is an interesting case: monthly (Sharpe 0.684) slightly outperforms weekly (0.678), though both substantially beat daily (0.531).
This may reflect AMZN's stronger trending behavior during this period.

\subsection{Statistical Analysis}
\label{sec:stats}

We conduct paired $t$-tests across the five stocks to test our hypotheses, with Cohen's $d$ effect sizes:

\para{Weekly vs.\ daily.}
The mean Sharpe difference is +0.136 in favor of weekly, with a medium effect size ($d = 0.59$).
The paired $t$-test yields $p = 0.256$, which does not reach significance at $\alpha = 0.05$.
The direction consistently supports the hypothesis (weekly better in 4/5 stocks), but the small sample size ($n = 5$) limits statistical power.

\para{Monthly vs.\ daily.}
The mean Sharpe difference is $-$0.472, indicating monthly is substantially worse, with a medium-to-large negative effect ($d = -0.66$).
The paired $t$-test yields $p = 0.213$.
While not significant at $\alpha = 0.05$ due to variance across stocks, the direction is unanimous: monthly underperforms daily for all five stocks on Sharpe ratio.

\subsection{Trade Efficiency}
\label{sec:efficiency}

\figref{fig:trades_return} plots the relationship between trade count and cumulative return.
Weekly rebalancing achieves higher returns with far fewer trades: 31 trades yield +62.4\% return, compared to 127 trades for +53.8\% (daily) and 11 trades for +13.5\% (monthly).
This implies that each weekly trade captures substantially more value than each daily trade, while monthly trades are too infrequent to capture available opportunities.
At 75\% fewer API calls, weekly rebalancing also reduces LLM inference costs proportionally.

\begin{figure}[t]
    \centering
    \begin{subfigure}[b]{0.48\textwidth}
        \includegraphics[width=\textwidth]{figures/trades_vs_return.png}
        \caption{Trade count vs.\ cumulative return.}
        \label{fig:trades_return}
    \end{subfigure}
    \hfill
    \begin{subfigure}[b]{0.48\textwidth}
        \includegraphics[width=\textwidth]{figures/sharpe_by_frequency.png}
        \caption{Sharpe ratio by strategy and frequency.}
        \label{fig:sharpe_bars}
    \end{subfigure}
    \caption{(\subref{fig:trades_return}) Weekly achieves the best return-per-trade ratio. (\subref{fig:sharpe_bars}) The LLM agent's Sharpe ratio peaks at weekly frequency, approaching \smacross performance.}
    \label{fig:efficiency}
\end{figure}

\subsection{Drawdown Analysis}
\label{sec:drawdown}

Contrary to our initial hypothesis that longer horizons would reduce drawdowns, we observe the opposite pattern (\figref{fig:sharpe_heatmap}).
Daily rebalancing achieves the lowest maximum drawdown ($-$20.7\%), followed by weekly ($-$24.2\%) and monthly ($-$28.0\%).
Frequent position adjustments allow the daily agent to exit losing positions quickly, providing better downside protection.
The monthly agent, deciding only 24 times over 2 years, cannot respond to intra-month corrections.
TSLA illustrates the extreme: daily MDD is $-$33.9\%, weekly $-$46.7\%, and monthly $-$48.1\%.

\begin{figure}[t]
    \centering
    \includegraphics[width=0.65\linewidth]{figures/sharpe_heatmap.png}
    \caption{Heatmap of Sharpe ratios by stock and frequency. The weekly column is consistently the brightest (highest values) except for TSLA. Monthly (right) shows sharp degradation for volatile stocks.}
    \label{fig:sharpe_heatmap}
\end{figure}
